
\documentclass{sig-alternate-05-2015}

\def\sharedaffiliation{%
\end{tabular}
\begin{tabular}{c}}

\begin{document}

\title{Tic-Tac-Tobot}
\subtitle{[EECS 149/249A Class Project]}


\numberofauthors{2}
\author{
% 1st. author
\alignauthor
Patrick Scheffe \\
\email{p.s@berkeley.edu} \\
% 2nd. author
\alignauthor
Nikolas Alberti \\
\email{nikolas.alberti@berkeley.edu} \\
% 3rd. author
\sharedaffiliation
\affaddr{Department of Electrical Engineering and Computer Science} \\
\affaddr{University of California}\\
\affaddr{Berkeley, CA} \\
}

\maketitle
\begin{abstract}

\end{abstract}
\section{Introduction}

\section{Hardware}

\section{Robotic Manipulator}
\subsection{Choosing the dimensions}
%TODO: citation Joo
For accurate two dimensional actuation, a classical robotic manipulator that  consists a chain of dependent joints is not suitable for a low budget approach. The plotclock, a project by Johannes "\emph{Joo}" Heberlein from the Fablab Nuremberg (comparable to a makerspace) impressively proved a way to make low budget 2D actuation work. Therefore, we planed to chose a similar manipulator. However, while the plotclock needs to cover a range whose horizontal dimensions exceed the vertical, it was not possible to blindly adapt the manipulator. For our project, actuation needs to be performed in an area formed like a square. Therefore, we needed to resize the limbs.

This problem can be described as a maximization problem:
 The desired area of the largest possible square that fits into the reachable range of the actuator is given. Find the dimensions $L_1$, $L_2$ and $L_3$ that minimize the sum $L_1 + L_2 +L3$.
 
 Unfortunately, the function of the largest square $A(L_1, L_2, L_3)$ is not linear and therefore the problem can not be solved by partly derive towards $L_1$, $L_2$ and $L_3$ and set the derivations to zero. 
 We therefore chose to model the reachable space in a geometry software called \emph{Geogebra}, an open source software for supporting mathematical education in schools. Then, we empirically derived a near optimal sizing that yields a reachable square of $100 cm^2$-
\subsection{Manufacturing}
\subsection{Simplified Kinematic Model}\label{section1}

A simplified sketch of the robotic manipulator can be seen in Figure \ref{fig:model}.
As a first approximation it is useful to determine the angles $\theta_1$ to $\theta_4$ from the given position of the joint at which the two arms coincide (x,y). The key for solving this inverse kinematics problem is to divide it into smaller subproblems that each can be solved individually. For that purpose, the line segments $a$, $b$ and $c$ are introduced.
\begin{figure}[!h]
	\label{fig:model}
	\centering
	\includegraphics[width=.42\textwidth]{LinkDiagramSimple_try.pdf}
	\caption{Simplified Version of the Manipulator}
\end{figure}

From the initial information, following values can be directly computed:
\begin{equation}
\alpha = arctan \left( \frac{y}{x} \right)
\end{equation}
\begin{equation}
\beta = \pi - \alpha
\end{equation}
\begin{equation}
c = \sqrt{x^2 + y^2}
\end{equation}
\begin{equation}
a = \sqrt{\left(x+ \frac{L_3}{2}\right)^2 + y^2}
\end{equation}
\begin{equation}
b = \sqrt{\left(x- \frac{L_3}{2}\right)^2 + y^2}
\end{equation}


Now, you can solve for $\theta_2$ and $\theta_3$ by either using sine rule or cosine rule. However, the sine rule can be ambiguous in certain setups, which makes case differentiation necessary. Although mathematically steady, in our implementation the domain crossing from one solution to the other resulted in discontinuities of the movement. Therefore, the cosine rule solution is preferred:
\begin{equation}
\theta_2 = arccos\left(  \frac{a^2 + (\frac{L_3}{2})^2 - c^2}{aL_3} \right)
\label{eqn:theta2}
\end{equation}
\begin{equation}
\theta_3 = arccos\left(  \frac{b^2 + (\frac{L_3}{2})^2 - c^2}{bL_3} \right)
\label{eqn:theta3}
\end{equation}


Using cosine rule we can also solve for $\theta_1$ and $\theta_4$:
\begin{equation}
\theta_1 = arccos\left(  \frac{a^2 + {L_1}^2 - {L_2}^2}{2aL_1} \right)
\label{eqn:theta1}
\end{equation}

\begin{equation}
\theta_4 = arccos\left(  \frac{b^2 + {L_1}^2 - {L_2}^2}{2bL_1} \right)
\label{eqn:theta4}
\end{equation}

\subsection{Complete Kinematic Model}
The simplified version of the kinematic model is good for quickly creating a working implementation. However, any movements executed by the manipulator will suffer from distortion. The pen is not mounted \emph{exactly} at the joint's position but in a small distance. Hence, a precise solution is necessary. For that purpose, new definitions must be made (Figure \ref{fig:model2}).
\begin{figure}[!h]
	\label{fig:model2}
	\centering
	\includegraphics[width=.42\textwidth]{LinkDiagramComplicated_try.pdf}
	\caption{Complete Version of the Manipulator}
\end{figure}

$L_4$ and $\epsilon$ are fixed measures and not influenced by the position of the manipulator:
\begin{equation}
L_4 = \sqrt{{L_2}^2 + {L_5}^2 - 2L_5L_2cos\left(\frac{3\pi}{4}\right)}
\label{eqn:L4}
\end{equation}
\begin{equation}
\epsilon = arccos\left(  \frac{{L_4}^2 + {L_2}^2 - {L_5}^2}{2L_4L_2} \right)
\label{eqn:tepsilon}
\end{equation}

$d$, $e$ and $f$ can be yielded by the Pythagorean theorem:
\begin{equation}
f = \sqrt{{x'}^2 + {y'}^2}
\label{eqn:f}
\end{equation}
\begin{equation}
d = \sqrt{\left(x'+ \frac{L_3}{2}\right)^2 + {y'}^2}
\end{equation}
\begin{equation}
e = \sqrt{\left(x'- \frac{L_3}{2}\right)^2 + {y'}^2}
\end{equation}
Now, $\theta_2'$, $\theta_3'$ and $\delta$ can be computed using cosine rule:
\begin{equation}
\theta_2' = arccos\left(  \frac{d^2 + (\frac{L_3}{2})^2 - f^2}{dL_3} \right)
\label{eqn:theta2prime}
\end{equation}
\begin{equation}
\theta_3' = arccos\left(  \frac{e^2 + (\frac{L_3}{2})^2 - f^2}{eL_3} \right)
\label{eqn:theta3prime}
\end{equation}
\begin{equation}
\delta = arccos\left(  \frac{{L_4}^2 + {L_1}^2 - d^2}{2L_4L_1} \right)
\label{eqn:delta}
\end{equation}
$\delta$ is the sum of $\epsilon$ and $\gamma$.
\begin{equation}
\gamma = \delta - \epsilon
\end{equation}
That allows us to calculate some quantities of the simple kinematics model:
\begin{equation}
a = \sqrt{{L_1}^2 + {L_1}^2 - 2L_1L_2cos\left(\gamma\right)}
\end{equation}
\begin{equation}
\theta_1 = arccos\left(  \frac{a^2 + {L_1}^2 - {L_2}^2}{2aL_1} \right)
\end{equation}
\begin{equation}
\theta_2 = \theta_1' + \theta_2' - \theta_1
\end{equation}
Finally, we are able to find the position of the joint at which the two arms coincide:
\begin{equation}
y = a \sin(\theta_2)
\end{equation}
\begin{equation}
x = a\cos(\theta_2) - \frac{L_3}{2}
\end{equation}
Now, the methods from section \ref{section1} can be used to solve for $\theta_3$ and $\theta_4$.

\section{Kinematics Using the Arduino}
\subsection{Modelling the Servo Motors}
Servo motors are motors that do not support continuous motion but in return precisely can be driven to a desired angle. They have three external cables in different colors. The black cable should be connected to ground and the red one to $V_{DD}$ (approximately 5V). These two cables supply the servo motor with the necessary power. The third cable (white, yellow or orange) carries the control signal. The control is done by pulse width modulation (PWM). The servo motor expects to receive a pulse every 20\,ms with a pulse width between 1\,ms and 2\,ms. By proportional control, the servo motor assumes its most positive position at the the pulse width of 1\,ms and the most negative position at a pulse width of 2\,ms as defined in the mathematical direction of rotation. The range in between these extrema can be assumed to be linearly covered, i.e. a pulse width of 1.5\,ms should yield the position in between these positions. A function can be derived that maps the pulse width to an angular position of the servomotor:
\begin{eqnarray}
Angle(t_{pulse}) &=& \frac{t_{pulse} -1 \text{ms} }{2 \text{ms}-1 \text{ms}}\cdot \left(Angle_{max}-Angle_{min}\right)&\nonumber \\ &&+Angle_{min},&1 \text{ms}\le t_{pulse}\le 2 \text{ms}  \nonumber
\end{eqnarray}
This is just a model of the movement of the servo motors. There are three causes that the actual behavior deviates from the model:
\begin{itemize}
	\item A high torque forces the servo motors from leaving its desired position.
	\item The backlash of the gears in the servo motors adds an inaccuracy to the position.
	\item Nonlinearities make the servomotor cover the range of movement not evenly.
\end{itemize}
Furthermore, when the pulse width modulation signal is created by a digital signal, a quantization error occurs.
As you can see, the model is making some approximations. The inner of the servo motor is treated as black box. However, adding the details would bloat the model and the gain is questionable. The proposed model is precise enough to be useful but not so complex that it becomes cumbersome.
\subsection{Software on the Arduino}
The Arduino Uno has six PWM pins available.
Very conveniently, the Arduino IDE already is equipped with a library \texttt{Servo}. This library allows us to instantiate objects of the type \texttt{Servo}. The most important functions on this object are \texttt{attach()} and \texttt{write()}. The \texttt{attach()} function assigns the Servo object to a GPIO pin. The \texttt{write()} maps an angle in the range of $0^\circ$ to $180^\circ$ to a pulse width and makes the attached GPIO pin assume the according PWM. Inherently, a reachable range of $180^\circ$ is assumed for the servo motor. However, our servo motors only are capable of spinning $150^\circ$. A mapping between an angle from $0^\circ$ to $150^\circ$ to an angle between $0^\circ$ to $180^\circ$ has to be performed.

Furthermore Zahnrad!
\section{Computer Vision}

\section{The Tic-Tac-Toe AI}

\newpage
\newpage
\newpage
\begin{abstract}
This paper provides a sample of a \LaTeX\ document which conforms,
somewhat loosely, to the formatting guidelines for
ACM SIG Proceedings. It is an {\em alternate} style which produces
a {\em tighter-looking} paper and was designed in response to
concerns expressed, by authors, over page-budgets.
It complements the document \textit{Author's (Alternate) Guide to
Preparing ACM SIG Proceedings Using \LaTeX$2_\epsilon$\ and Bib\TeX}.
This source file has been written with the intention of being
compiled under \LaTeX$2_\epsilon$\ and BibTeX.

The developers have tried to include every imaginable sort
of ``bells and whistles", such as a subtitle, footnotes on
title, subtitle and authors, as well as in the text, and
every optional component (e.g. Acknowledgments, Additional
Authors, Appendices), not to mention examples of
equations, theorems, tables and figures.

To make best use of this sample document, run it through \LaTeX\
and BibTeX, and compare this source code with the printed
output produced by the dvi file. A compiled PDF version
is available on the web page to help you with the
`look and feel'.
\end{abstract}


\section{Introduction}
The \textit{proceedings} are the records of a conference.
ACM seeks to give these conference by-products a uniform,
high-quality appearance.  To do this, ACM has some rigid
requirements for the format of the proceedings documents: there
is a specified format (balanced  double columns), a specified
set of fonts (Arial or Helvetica and Times Roman) in
certain specified sizes (for instance, 9 point for body copy),
a specified live area (18 $\times$ 23.5 cm [7" $\times$ 9.25"]) centered on
the page, specified size of margins (1.9 cm [0.75"]) top, (2.54 cm [1"]) bottom
and (1.9 cm [.75"]) left and right; specified column width
(8.45 cm [3.33"]) and gutter size (.83 cm [.33"]).

The good news is, with only a handful of manual
settings\footnote{Two of these, the {\texttt{\char'134 numberofauthors}}
and {\texttt{\char'134 alignauthor}} commands, you have
already used; another, {\texttt{\char'134 balancecolumns}}, will
be used in your very last run of \LaTeX\ to ensure
balanced column heights on the last page.}, the \LaTeX\ document
class file handles all of this for you.

The remainder of this document is concerned with showing, in
the context of an ``actual'' document, the \LaTeX\ commands
specifically available for denoting the structure of a
proceedings paper, rather than with giving rigorous descriptions
or explanations of such commands.

\section{The {\secit Body} of The Paper}
Typically, the body of a paper is organized
into a hierarchical structure, with numbered or unnumbered
headings for sections, subsections, sub-subsections, and even
smaller sections.  The command \texttt{{\char'134}section} that
precedes this paragraph is part of such a
hierarchy.\footnote{This is the second footnote.  It
starts a series of three footnotes that add nothing
informational, but just give an idea of how footnotes work
and look. It is a wordy one, just so you see
how a longish one plays out.} \LaTeX\ handles the numbering
and placement of these headings for you, when you use
the appropriate heading commands around the titles
of the headings.  If you want a sub-subsection or
smaller part to be unnumbered in your output, simply append an
asterisk to the command name.  Examples of both
numbered and unnumbered headings will appear throughout the
balance of this sample document.

Because the entire article is contained in
the \textbf{document} environment, you can indicate the
start of a new paragraph with a blank line in your
input file; that is why this sentence forms a separate paragraph.

\subsection{Type Changes and {\subsecit Special} Characters}
We have already seen several typeface changes in this sample.  You
can indicate italicized words or phrases in your text with
the command \texttt{{\char'134}textit}; emboldening with the
command \texttt{{\char'134}textbf}
and typewriter-style (for instance, for computer code) with
\texttt{{\char'134}texttt}.  But remember, you do not
have to indicate typestyle changes when such changes are
part of the \textit{structural} elements of your
article; for instance, the heading of this subsection will
be in a sans serif\footnote{A third footnote, here.
Let's make this a rather short one to
see how it looks.} typeface, but that is handled by the
document class file. Take care with the use
of\footnote{A fourth, and last, footnote.}
the curly braces in typeface changes; they mark
the beginning and end of
the text that is to be in the different typeface.

You can use whatever symbols, accented characters, or
non-English characters you need anywhere in your document;
you can find a complete list of what is
available in the \textit{\LaTeX\
User's Guide}\cite{Lamport:LaTeX}.

\subsection{Math Equations}
You may want to display math equations in three distinct styles:
inline, numbered or non-numbered display.  Each of
the three are discussed in the next sections.

\subsubsection{Inline (In-text) Equations}
A formula that appears in the running text is called an
inline or in-text formula.  It is produced by the
\textbf{math} environment, which can be
invoked with the usual \texttt{{\char'134}begin. . .{\char'134}end}
construction or with the short form \texttt{\$. . .\$}. You
can use any of the symbols and structures,
from $\alpha$ to $\omega$, available in
\LaTeX\cite{Lamport:LaTeX}; this section will simply show a
few examples of in-text equations in context. Notice how
this equation: \begin{math}\lim_{n\rightarrow \infty}x=0\end{math},
set here in in-line math style, looks slightly different when
set in display style.  (See next section).

\subsubsection{Display Equations}
A numbered display equation -- one set off by vertical space
from the text and centered horizontally -- is produced
by the \textbf{equation} environment. An unnumbered display
equation is produced by the \textbf{displaymath} environment.

Again, in either environment, you can use any of the symbols
and structures available in \LaTeX; this section will just
give a couple of examples of display equations in context.
First, consider the equation, shown as an inline equation above:
\begin{equation}\lim_{n\rightarrow \infty}x=0\end{equation}
Notice how it is formatted somewhat differently in
the \textbf{displaymath}
environment.  Now, we'll enter an unnumbered equation:
\begin{displaymath}\sum_{i=0}^{\infty} x + 1\end{displaymath}
and follow it with another numbered equation:
\begin{equation}\sum_{i=0}^{\infty}x_i=\int_{0}^{\pi+2} f\end{equation}
just to demonstrate \LaTeX's able handling of numbering.

\subsection{Citations}
Citations to articles \cite{bowman:reasoning,
clark:pct, braams:babel, herlihy:methodology},
conference proceedings \cite{clark:pct} or
books \cite{salas:calculus, Lamport:LaTeX} listed
in the Bibliography section of your
article will occur throughout the text of your article.
You should use BibTeX to automatically produce this bibliography;
you simply need to insert one of several citation commands with
a key of the item cited in the proper location in
the \texttt{.tex} file \cite{Lamport:LaTeX}.
The key is a short reference you invent to uniquely
identify each work; in this sample document, the key is
the first author's surname and a
word from the title.  This identifying key is included
with each item in the \texttt{.bib} file for your article.

The details of the construction of the \texttt{.bib} file
are beyond the scope of this sample document, but more
information can be found in the \textit{Author's Guide},
and exhaustive details in the \textit{\LaTeX\ User's
Guide}\cite{Lamport:LaTeX}.

This article shows only the plainest form
of the citation command, using \texttt{{\char'134}cite}.
This is what is stipulated in the SIGS style specifications.
No other citation format is endorsed or supported.

\subsection{Tables}
Because tables cannot be split across pages, the best
placement for them is typically the top of the page
nearest their initial cite.  To
ensure this proper ``floating'' placement of tables, use the
environment \textbf{table} to enclose the table's contents and
the table caption.  The contents of the table itself must go
in the \textbf{tabular} environment, to
be aligned properly in rows and columns, with the desired
horizontal and vertical rules.  Again, detailed instructions
on \textbf{tabular} material
is found in the \textit{\LaTeX\ User's Guide}.

Immediately following this sentence is the point at which
Table 1 is included in the input file; compare the
placement of the table here with the table in the printed
dvi output of this document.

\begin{table}
\centering
\caption{Frequency of Special Characters}
\begin{tabular}{|c|c|l|} \hline
Non-English or Math&Frequency&Comments\\ \hline
\O & 1 in 1,000& For Swedish names\\ \hline
$\pi$ & 1 in 5& Common in math\\ \hline
\$ & 4 in 5 & Used in business\\ \hline
$\Psi^2_1$ & 1 in 40,000& Unexplained usage\\
\hline\end{tabular}
\end{table}

To set a wider table, which takes up the whole width of
the page's live area, use the environment
\textbf{table*} to enclose the table's contents and
the table caption.  As with a single-column table, this wide
table will ``float" to a location deemed more desirable.
Immediately following this sentence is the point at which
Table 2 is included in the input file; again, it is
instructive to compare the placement of the
table here with the table in the printed dvi
output of this document.


\begin{table*}
\centering
\caption{Some Typical Commands}
\begin{tabular}{|c|c|l|} \hline
Command&A Number&Comments\\ \hline
\texttt{{\char'134}alignauthor} & 100& Author alignment\\ \hline
\texttt{{\char'134}numberofauthors}& 200& Author enumeration\\ \hline
\texttt{{\char'134}table}& 300 & For tables\\ \hline
\texttt{{\char'134}table*}& 400& For wider tables\\ \hline\end{tabular}
\end{table*}
% end the environment with {table*}, NOTE not {table}!

\subsection{Figures}
Like tables, figures cannot be split across pages; the
best placement for them
is typically the top or the bottom of the page nearest
their initial cite.  To ensure this proper ``floating'' placement
of figures, use the environment
\textbf{figure} to enclose the figure and its caption.

This sample document contains examples of \textbf{.eps} files to be
displayable with \LaTeX.  If you work with pdf\LaTeX, use files in the
\textbf{.pdf} format.  Note that most modern \TeX\ system will convert
\textbf{.eps} to \textbf{.pdf} for you on the fly.  More details on
each of these is found in the \textit{Author's Guide}.

\begin{figure}
\centering
\includegraphics{fly}
\caption{A sample black and white graphic.}
\end{figure}

\begin{figure}
\centering
\includegraphics[height=1in, width=1in]{fly}
\caption{A sample black and white graphic
that has been resized with the \texttt{includegraphics} command.}
\end{figure}


As was the case with tables, you may want a figure
that spans two columns.  To do this, and still to
ensure proper ``floating'' placement of tables, use the environment
\textbf{figure*} to enclose the figure and its caption.
and don't forget to end the environment with
{figure*}, not {figure}!

\begin{figure*}
\centering
\includegraphics{flies}
\caption{A sample black and white graphic
that needs to span two columns of text.}
\end{figure*}


\begin{figure}
\centering
\includegraphics[height=1in, width=1in]{rosette}
\caption{A sample black and white graphic that has
been resized with the \texttt{includegraphics} command.}
\vskip -6pt
\end{figure}

\subsection{Theorem-like Constructs}
Other common constructs that may occur in your article are
the forms for logical constructs like theorems, axioms,
corollaries and proofs.  There are
two forms, one produced by the
command \texttt{{\char'134}newtheorem} and the
other by the command \texttt{{\char'134}newdef}; perhaps
the clearest and easiest way to distinguish them is
to compare the two in the output of this sample document:

This uses the \textbf{theorem} environment, created by
the\linebreak\texttt{{\char'134}newtheorem} command:
\newtheorem{theorem}{Theorem}
\begin{theorem}
Let $f$ be continuous on $[a,b]$.  If $G$ is
an antiderivative for $f$ on $[a,b]$, then
\begin{displaymath}\int^b_af(t)dt = G(b) - G(a).\end{displaymath}
\end{theorem}

The other uses the \textbf{definition} environment, created
by the \texttt{{\char'134}newdef} command:
\newdef{definition}{Definition}
\begin{definition}
If $z$ is irrational, then by $e^z$ we mean the
unique number which has
logarithm $z$: \begin{displaymath}{\log e^z = z}\end{displaymath}
\end{definition}

Two lists of constructs that use one of these
forms is given in the
\textit{Author's  Guidelines}.
 
There is one other similar construct environment, which is
already set up
for you; i.e. you must \textit{not} use
a \texttt{{\char'134}newdef} command to
create it: the \textbf{proof} environment.  Here
is a example of its use:
\begin{proof}
Suppose on the contrary there exists a real number $L$ such that
\begin{displaymath}
\lim_{x\rightarrow\infty} \frac{f(x)}{g(x)} = L.
\end{displaymath}
Then
\begin{displaymath}
l=\lim_{x\rightarrow c} f(x)
= \lim_{x\rightarrow c}
\left[ g{x} \cdot \frac{f(x)}{g(x)} \right ]
= \lim_{x\rightarrow c} g(x) \cdot \lim_{x\rightarrow c}
\frac{f(x)}{g(x)} = 0\cdot L = 0,
\end{displaymath}
which contradicts our assumption that $l\neq 0$.
\end{proof}

Complete rules about using these environments and using the
two different creation commands are in the
\textit{Author's Guide}; please consult it for more
detailed instructions.  If you need to use another construct,
not listed therein, which you want to have the same
formatting as the Theorem
or the Definition\cite{salas:calculus} shown above,
use the \texttt{{\char'134}newtheorem} or the
\texttt{{\char'134}newdef} command,
respectively, to create it.

\subsection*{A {\secit Caveat} for the \TeX\ Expert}
Because you have just been given permission to
use the \texttt{{\char'134}newdef} command to create a
new form, you might think you can
use \TeX's \texttt{{\char'134}def} to create a
new command: \textit{Please refrain from doing this!}
Remember that your \LaTeX\ source code is primarily intended
to create camera-ready copy, but may be converted
to other forms -- e.g. HTML. If you inadvertently omit
some or all of the \texttt{{\char'134}def}s recompilation will
be, to say the least, problematic.

\section{Conclusions}
This paragraph will end the body of this sample document.
Remember that you might still have Acknowledgments or
Appendices; brief samples of these
follow.  There is still the Bibliography to deal with; and
we will make a disclaimer about that here: with the exception
of the reference to the \LaTeX\ book, the citations in
this paper are to articles which have nothing to
do with the present subject and are used as
examples only.
%\end{document}  % This is where a 'short' article might terminate

%ACKNOWLEDGMENTS are optional
\section{Acknowledgments}
This section is optional; it is a location for you
to acknowledge grants, funding, editing assistance and
what have you.  In the present case, for example, the
authors would like to thank Gerald Murray of ACM for
his help in codifying this \textit{Author's Guide}
and the \textbf{.cls} and \textbf{.tex} files that it describes.

%
% The following two commands are all you need in the
% initial runs of your .tex file to
% produce the bibliography for the citations in your paper.
\bibliographystyle{abbrv}
\bibliography{sigproc}  % sigproc.bib is the name of the Bibliography in this case
% You must have a proper ".bib" file
%  and remember to run:
% latex bibtex latex latex
% to resolve all references
%
% ACM needs 'a single self-contained file'!
%
%APPENDICES are optional
%\balancecolumns
\appendix
%Appendix A
\section{Headings in Appendices}
The rules about hierarchical headings discussed above for
the body of the article are different in the appendices.
In the \textbf{appendix} environment, the command
\textbf{section} is used to
indicate the start of each Appendix, with alphabetic order
designation (i.e. the first is A, the second B, etc.) and
a title (if you include one).  So, if you need
hierarchical structure
\textit{within} an Appendix, start with \textbf{subsection} as the
highest level. Here is an outline of the body of this
document in Appendix-appropriate form:
\subsection{Introduction}
\subsection{The Body of the Paper}
\subsubsection{Type Changes and  Special Characters}
\subsubsection{Math Equations}
\paragraph{Inline (In-text) Equations}
\paragraph{Display Equations}
\subsubsection{Citations}
\subsubsection{Tables}
\subsubsection{Figures}
\subsubsection{Theorem-like Constructs}
\subsubsection*{A Caveat for the \TeX\ Expert}
\subsection{Conclusions}
\subsection{Acknowledgments}
\subsection{Additional Authors}
This section is inserted by \LaTeX; you do not insert it.
You just add the names and information in the
\texttt{{\char'134}additionalauthors} command at the start
of the document.
\subsection{References}
Generated by bibtex from your ~.bib file.  Run latex,
then bibtex, then latex twice (to resolve references)
to create the ~.bbl file.  Insert that ~.bbl file into
the .tex source file and comment out
the command \texttt{{\char'134}thebibliography}.
% This next section command marks the start of
% Appendix B, and does not continue the present hierarchy
\section{More Help for the Hardy}
The sig-alternate.cls file itself is chock-full of succinct
and helpful comments.  If you consider yourself a moderately
experienced to expert user of \LaTeX, you may find reading
it useful but please remember not to change it.
%\balancecolumns % GM June 2007
% That's all folks!
\end{document}